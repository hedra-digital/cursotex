
%!TEX program = xelatex
\documentclass{memoir}

\usepackage[polutonikogreek,english,italian,ngerman,french,russian,brazilian]{babel}
\usepackage[protrusion=true,final]{microtype}
\usepackage{fontspec}

\usepackage[table,xcdraw]{xcolor}


% Pacotes importantes
\usepackage{microtype}
\usepackage{graphicx,lipsum,verse}

% Hifenização
\hyphenation{Schniz-el}

% Trocando de fonte
\setmainfont{Times New Roman}
 

% headings
\makeevenhead{headings}{A}{B}{C}
\makeoddhead{headings}{C}{B}{A}
\makeevenfoot{headings}{A}{B}{C}
\makeoddfoot{headings}{C}{B}{A}

\title{Exemplo de abertura de artigo}
\author{Jorge Sallum}
 


\begin{document}
 

\maketitle
 

\setcounter{secnumdepth}{4} 
\tableofcontents


%\parskip
%\parindent

\chapter[Alice no país...]{Alice no país das maravilhas}

\epigraph{Lorem ipsum dolor sit amet, consetetur sadipscing elitr, sed diam nonumy eirmod tempor invidunt ut labore et dolore magna aliquyam erat, sed diam voluptua.}{Benjamim Franklin}

% Break, par, \\
\section{Seção}
\lipsum[1]

\subsection{Subseção}

\begin{verse}
Batatinha quando nasce espalha a rama pelo chão.\\
menininha quando dorme põe a mão no coração.\\
Sou pequenininha do tamanho de um botão,\\
carrego papai no bolso e mamãe no coração\\
O bolso furou e o papai caiu no chão.\\*
Mamãe que é mais querida ficou no coração.\\!

Batatinha quando nasce espalha a rama pelo chão.\\
menininha quando dorme põe a mão no coração.\\
Sou pequenininha do tamanho de um botão,\\
carrego papai no bolso e mamãe no coração\\
O bolso furou e o papai caiu no chão.\\*
Mamãe que é mais querida ficou no coração.\\*
\nobreak\medskip
\hfill J.S.~Bach

\end{verse}
\lipsum[2]


\subsubsection{Subsubseção}
\lipsum[3]

\paragraph{Parágrafo}
\lipsum[4]

\begin{figure}[t]
\caption{Gato em linhas}
\includegraphics[width=6cm]{cat.jpg}
\caption{Este é um gato em linhas}
\end{figure}

\subparagraph{Subparágrafo}
\lipsum[5]


% Please add the following required packages to your document preamble:
% \usepackage[table,xcdraw]{xcolor}
% If you use beamer only pass "xcolor=table" option, i.e. \documentclass[xcolor=table]{beamer}
\begin{table}[b]
\centering
\caption{Esta é uma tabela}
\label{my-label}
\begin{tabular}{rl}
\rowcolor[HTML]{ECF4FF} 
\textbf{ACNUR} & \textit{Alto Comissariado das Nações Unidas para os Refugiados} \\
\textbf{MSF}   & \textit{Médicos Sem Fronteiras}                                 \\
\rowcolor[HTML]{ECF4FF} 
\textbf{MdM}   & \textit{Médicos do Mundo}                                       \\
\textbf{OIM}   & \textit{Organização Internacional para as Migrações}            \\
\rowcolor[HTML]{ECF4FF} 
\textbf{ONG}   & \textit{Organização Não"-Governamental}                         \\
\textbf{ONU}   & \textit{Organização das Nações Unidas}                          \\
\rowcolor[HTML]{ECF4FF} 
\textbf{UE}    & \textit{União Europeia}                                        
\end{tabular}
\end{table}


\lipsum

\noindent\dotfill\par

%\noindent\rule{\textwidth}{3pt}

\end{document}
