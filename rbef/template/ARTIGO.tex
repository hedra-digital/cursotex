
	\begin{abstract}
Um abstract em vernáculo. \lipsum[1].
    \palavraschave{xxxxx}

	\end{abstract}

	
	

	\begin{abstract}
\lipsum[1].
	\keywords{xxxx}
	
	\end{abstract}

\saythanks

\begin{comment}

\section{Uma seção qualquer}

\lipsum[1]

\section{Uma seção qualquer: 2}

\lipsum[1]

\begin{widetext}

  \begin{displaymath}
a + b + c + d + e + f + g + h
\end{displaymath}


\end{widetext}


\[
  \dfrac{q}{m} \propto \frac{4\pi{r^2}}{\frac{4}{3}\pi{r^3}}\
  \propto 3{r^\alpha}\text{, \quad gdje je\quad}\;\alpha = -1 \text{.}
\]

\lipsum[1-2]
Fórmula
%$\int\frac{1}{2}dx - \mathlarger{ \int\frac{1}{2}dx}$
\lipsum[1-20]

\section{Você pode utilizar os ambiente \texttt{multicols}}
Se tiver muitas equações longas. Simplesmente utilize 
\begin{verbatim} 
\comeco
\end{verbatim}

Para começar o texto depois do abstract.

Ou então

\begin{verbatim}
\duascolunas
\end{verbatim}


Da mesma forma, use
\begin{verbatim} 
\fim
\end{verbatim}

(Preferivelmente no \verb+fim+ do documento\ldots)

Ou então

\begin{verbatim}
\umacoluna
\end{verbatim}

Para passar para uma coluna e começar o ambiente \verb+widetext+:

\begin{verbatim}
\umacoluna % ou \end{multicols}
\begin{widetext}
\begin{equation}
a + b^2 = 3
\end{equation}
\end{widetext}
\duascolunas % ou \begin{multicols}{2}
...
\fim % ou \end{multicols}


\end{verbatim}

\end{comment}
